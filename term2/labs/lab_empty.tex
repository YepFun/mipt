\documentclass[a4paper,12pt]{article} % тип документа

% Поля страниц
\usepackage[left=2.5cm,right=2.5cm,
    top=2cm,bottom=2cm,bindingoffset=0cm]{geometry}
    
%Пакет дял таблиц   
\usepackage{multirow} 
    
%Отступ после заголовка    
\usepackage{indentfirst}


% Рисунки
\usepackage{floatrow,graphicx,calc}
\usepackage{wrapfig}

%%% Работа с картинками
\usepackage{graphicx}  % Для вставки рисунков
\graphicspath{{images/}}  % папки с картинками
\setlength\fboxsep{3pt} % Отступ рамки \fbox{} от рисунка
\setlength\fboxrule{1pt} % Толщина линий рамки \fbox{}
\usepackage{wrapfig} % Обтекание рисунков и таблиц текстом

% Создаём новый разделитель
\DeclareFloatSeparators{mysep}{\hspace{1cm}}

% Ссылки?
\usepackage{hyperref}
\usepackage[rgb]{xcolor}
\hypersetup{				% Гиперссылки
    colorlinks=true,       	% false: ссылки в рамках
	urlcolor=blue          % на URL
}


%  Русский язык
\usepackage[T2A]{fontenc}			% кодировка
\usepackage[utf8]{inputenc}			% кодировка исходного текста
\usepackage[english,russian]{babel}	% локализация и переносы

% Математика
\usepackage{amsmath,amsfonts,amssymb,amsthm,mathtools}

%%% Дополнительная работа с математикой
\usepackage{amsmath,amsfonts,amssymb,amsthm,mathtools} % AMS
\usepackage{icomma} % "Умная" запятая: $0,2$ --- число, $0, 2$ --- перечисление

% Что-то 
\usepackage{wasysym}


\begin{document}
\begin{center}
	\footnotesize{ФЕДЕРАЛЬНОЕ ГОСУДАРСТВЕННОЕ АВТОНОМНОЕ ОБРАЗОВАТЕЛЬНОЕ 			УЧРЕЖДЕНИЕ ВЫСШЕГО ОБРАЗОВАНИЯ}\\
	\footnotesize{МОСКОВСКИЙ ФИЗИКО-ТЕХНИЧЕСКИЙ ИНСТИТУТ\\(НАЦИОНАЛЬНЫЙ 			ИССЛЕДОВАТЕЛЬСКИЙ УНИВЕРСИТЕТ)}\\
	\footnotesize{ФИЗТЕХ-ШКОЛА ФИЗИКИ И ИССЛЕДОВАНИЙ им. ЛАНДАУ\\}
	\hfill \break
	\hfill \break
	\hfill \break
	\hfill \break
\end{center}

\begin{center}   
    \hfill \break
	\hfill \break
	\hfill \break
	\hfill \break    \hfill \break
	\hfill \break
	\hfill \break
	\hfill \break
    \hfill \break
    \hfill \break
	\hfill \break
	\large{Лабораторная работа № \\\textbf{}}\\
	\begin{flushright}
		Плотникова Анастасия Александровна\\
		Группа Б02-406
	\end{flushright}
	\hfill \break
	\hfill \break
	\hfill \break
\end{center}
\hfill \break
\hfill \break
\hfill \break
\hfill \break
\hfill \break
\hfill \break
\hfill \break
\hfill \break
\hfill \break
\hfill \break
\hfill \break
\hfill \break
\hfill \break
\begin{center}
	Долгопрудный, 2025 г.
\end{center}
\thispagestyle{empty}
\newpage
	\textbf{Цель работы:}\\ 
	\hfill \break
	
	\textbf{В работе используются:}\\ 
	
\section{Теоретическая справка}

\section{Экспериментальная установка}

Схема установки изображена на рисунке (\ref{}). 

\section{Ход эксперимента}

\begin{enumerate}
  \item
\end{enumerate}

\section{Результаты измерений}

\begin{table}[h]
  \caption{}
  \begin{tabular}{|c|c|c|}
      \hline $T$, $^\circ C$  &  $\Delta h$, мм & $P$, Па \\
      \hline  &  &  \\
      \hline  &  &  \\
      \hline  &  &  \\
      \hline  &  &  \\
      \hline  &  &  \\
      \hline  &  &  \\
      \hline  &  &  \\
      \hline  &  &  \\
      \hline  &  &  \\
      \hline  &  &  \\
      \hline 
  \end{tabular}
  \label{tab:}
\end{table}

\section{Вывод}
\end{document}
