\documentclass[a4paper,12pt]{article} % тип документа

% Поля страниц
\usepackage[left=2.5cm,right=2.5cm,top=2cm,bottom=2cm,bindingoffset=0cm]{geometry}

%Пакет для таблиц   
\usepackage{multirow} 

%Отступ после заголовка    
\usepackage{indentfirst}

% Рисунки
\usepackage{floatrow,graphicx,calc}
\usepackage{wrapfig}

%%% Работа с картинками
\usepackage{graphicx}  
\graphicspath{{images/}} 
\setlength\fboxsep{3pt}
\setlength\fboxrule{1pt}
\usepackage{wrapfig}

\DeclareFloatSeparators{mysep}{\hspace{1cm}}

\usepackage{hyperref}
\usepackage[rgb]{xcolor}
\hypersetup{
    colorlinks=true,
	urlcolor=blue
}

%  Русский язык
\usepackage[T2A]{fontenc}
\usepackage[utf8]{inputenc}
\usepackage[english,russian]{babel}

% Математика
\usepackage{amsmath,amsfonts,amssymb,amsthm,mathtools}
\usepackage{icomma}
\usepackage{wasysym}

\begin{document}
\begin{center}
	\footnotesize{ФЕДЕРАЛЬНОЕ ГОСУДАРСТВЕННОЕ АВТОНОМНОЕ ОБРАЗОВАТЕЛЬНОЕ УЧРЕЖДЕНИЕ ВЫСШЕГО ОБРАЗОВАНИЯ}\\
	\footnotesize{МОСКОВСКИЙ ФИЗИКО-ТЕХНИЧЕСКИЙ ИНСТИТУТ (НАЦИОНАЛЬНЫЙ ИССЛЕДОВАТЕЛЬСКИЙ УНИВЕРСИТЕТ)}\\
	\footnotesize{ФИЗТЕХ-ШКОЛА ФИЗИКИ И ИССЛЕДОВАНИЙ им. ЛАНДАУ}
\end{center}

\vspace{2cm}

\begin{center}
	\Large{\textbf{Лабораторная работа №2.1.4\\[0.5em]Определение теплоёмкости твёрдых тел}}
	
	\vspace{1cm}
	\begin{flushright}
		Плотникова Анастасия Александровна\\
		Группа Б02-406
	\end{flushright}
\end{center}

\vfill

\begin{center}
	Долгопрудный, 2025 г.
\end{center}

\newpage

\section*{Цель работы}
\begin{enumerate}
	\item Прямое измерение кривых нагревания $T_{\text{heat}}(t)$ и охлаждения $T_{\text{cool}}(t)$ пустого калориметра и системы «калориметр + твердое тело».
	\item Определение коэффициента теплоотдачи стенок калориметра.
	\item Определение теплоемкости пустого калориметра и удельной теплоемкости твердого тела.
\end{enumerate}

\section*{Оборудование}
Калориметр с нагревателем и термометром сопротивления, универсальные вольтметры В7-78/2 и В7-78/3, источник питания GPS-72303, вольтметр KEITHLEY, термопара K-типа, программа АКИП.

\section*{Теоретическая справка}

При подведении к телу количества тепла $\Delta Q$ за время $\Delta t$, изменение температуры $\Delta T$ связано с теплоёмкостью $C$ выражением:
\[
C = \frac{\Delta Q}{\Delta T}
\]

С учётом теплопотерь:
\[
C \Delta T = P \Delta t - \lambda (T - T_k) \Delta t \tag{2}
\]

В дифференциальной форме уравнения нагревания и охлаждения:
\[
C \frac{dT_{\text{heat}}(t)}{dt} = P - \lambda (T_{\text{heat}}(t) - T_k(t)) \tag{3}
\]
\[
C \frac{dT_{\text{cool}}(t)}{dt} = -\lambda (T_{\text{cool}}(t) - T_k(t)) \tag{4}
\]

Связь между сопротивлением термометра и температурой:
\[
R_T = R_{273} \left[ 1 + \alpha (T - 273) \right] \tag{5}
\]

Для пересчета сопротивлений:
\[
T(R_T) = \frac{R_T}{R_k} \cdot \frac{1 + \alpha (T_k - 273)}{1 + \alpha (T - 273)} \cdot 273 \tag{7}
\]

Для охлаждения (при $P=0$), интегрирование даёт:
\[
T(t) = (T_0 - T_k) e^{-\frac{\lambda}{C}t} + T_k \tag{11}
\]

По графику $\ln(T - T_k)$ от $t$ можно найти $\frac{\lambda}{C}$ — наклон прямой.

Для нагрева ($P \neq 0$):
\[
T(t) = \left(1 - e^{-\frac{\lambda}{C}t} \right) \cdot \frac{P}{\lambda} + T_k \tag{15}
\]

В точке, где $T = T_k$:
\[
C = \frac{P}{\left| \frac{dT_{\text{heat}}}{dt} \right|_{T = T_k}} \tag{16}
\]

\section*{Экспериментальная установка}

Экспериментальная установка включает калориметр с пенопластовой теплоизоляцией, помещённый в деревянный корпус. Внутренние стенки калориметра выполнены из материала с высокой теплопроводностью и имеют форму усечённых конусов, обеспечивая хороший тепловой контакт с телом.

Внутри калориметра установлены нагревательная спираль (СН) и термометр сопротивления. Температура измеряется через терморезистор, сопротивление которого зависит от температуры по формуле (5). Для измерений используются: 
\begin{itemize}
	\item универсальные вольтметры В7-78/2 и В7-78/3,
	\item вольтметр KEITHLEY,
	\item термопара K-типа,
	\item источник питания GPS-72303,
	\item программа АКИП B7-78 PT-Tool для автоматической записи данных.
\end{itemize}

\section*{Методика эксперимента}

\begin{enumerate}
	\item Включим в сеть все измерительные приборы и установим требуемые режимы измерений:
	\begin{itemize}
		\item Вольтметр В7-78/2 переведём в режим измерения температуры с термопарой K-типа.
		\item Один из вольтметров В7-78/3 настроим как омметр (четырёхпроводная схема).
		\item Второй В7-78/3 используем как амперметр.
		\item KEITHLEY автоматически перейдёт в режим измерения напряжения.
	\end{itemize}

	\item Запустим программу АКИП и проверим связь с приборами. Установим настройки сбора данных: скорость, количество точек, смещение и масштаб. Отметим нужные параметры в лабораторном журнале.

	\item Нажмём «Старт» в программе и будем вести непрерывную запись температур и сопротивлений.

	\item Охладим калориметр на $2\text{–}5\,^\circ\text{C}$ ниже комнатной температуры с помощью охлаждённого латунного образца. Через 3–4 минуты извлечём образец и подождём ещё 3–4 минуты, чтобы температура стабилизировалась.

	\item Измерим зависимость сопротивления терморезистора от времени $R_{\text{heat}}(t)$ при нагревании пустого калориметра:
	\begin{itemize}
		\item Включим нагреватель (кнопка «OUTPUT» на GPS-72303).
		\item Будем наблюдать за ростом температуры до превышения комнатной на $8\text{–}9\,^\circ\text{C}$.
		\item Отключим нагреватель и начнём запись охлаждения.
	\end{itemize}

	\item Измерим кривую охлаждения $R_{\text{cool}}(t)$ при $P = 0$.

	\item Повторим измерения нагревания и охлаждения для калориметра с телами (железо, алюминий). Перед каждым измерением охладим калориметр как в п. 4.

	\item После завершения измерений нажмём кнопку «Стоп» в программе и сохраним CSV-файлы с результатами измерений.

\end{enumerate}

\section*{Обработка результатов измерений}

\begin{enumerate}
	\item Откроем сохранённые CSV-файлы Record1 и Record2 в Excel или аналогичной программе. Каждый файл содержит:
	\begin{itemize}
		\item первую колонку — время в формате hh:mm:ss,
		\item вторую колонку — показания: температура в ${}^\circ$C (термопара) или сопротивление в Ом (терморезистор).
	\end{itemize}

	\item Преобразуем время в секунды от начала измерения:
	\begin{itemize}
		\item заменим значения времени на числовую последовательность от 0 с шагом 1 с.
	\end{itemize}

	\item Пересчитаем значения сопротивления терморезистора $R_T$ в температуру $T$ по калибровочной зависимости:
	\[
	T(R_T) = a R_T + b
	\]
	где:
	\[
	\text{Для установки 1: } T(R_T) = 14.584 \cdot R_T + 39.355 \\
	\text{Для установки 2: } T(R_T) = 14.378 \cdot R_T + 39.355
	\]

	\item Пересчитаем показания комнатной температуры $T_k$ в Кельвины:
	\[
	T_k(\text{K}) = T_k(^{\circ}C) + 273.15
	\]

	\item Построим графики $T_{\text{heat}}(t)$, $T_{\text{cool}}(t)$ и $T_k(t)$ на одном поле. По времени, зафиксированному в лабораторном журнале, определим участки соответствующие пустому калориметру, образцам из железа и алюминия.

	\item Построим график охлаждения в координатах:
	\[
	\left( \ln(T_{\text{cool}}(t) - T_k), \, t \right)
	\]
	По углу наклона прямой на линейном участке определим отношение:
	\[
	\frac{\lambda}{C}
	\]

	\item По формуле:
	\[
	T(t) = (T_0 - T_k) e^{-\frac{\lambda}{C}t} + T_k
	\]
	найдём $\lambda$ и затем определим теплоёмкость пустого калориметра:
	\[
	C = \frac{\lambda}{k}
	\]

	\item Повторим те же действия для калориметра с образцами из железа и алюминия. Получим полные теплоёмкости $C_\text{Fe+cal}$ и $C_\text{Al+cal}$. Тогда теплоёмкости образцов:
	\[
	C_\text{Fe} = C_\text{Fe+cal} - C_\text{cal}, \quad C_\text{Al} = C_\text{Al+cal} - C_\text{cal}
	\]

	\item Дополнительно определим $C$ и $\lambda$ дифференциальным методом:
	\begin{itemize}
		\item По формуле при $T = T_k$:
		\[
		C = \frac{P}{\left. \frac{dT_{\text{heat}}}{dt} \right|_{T = T_k}} \tag{16}
		\]

		\item Либо на одинаковых температурах $T$ при нагреве и охлаждении:
		\[
		C = \frac{P \left( \frac{dT}{dt} \big|_{\text{cool}} \right)}{\frac{dT}{dt} \big|_{\text{heat}} - \frac{dT}{dt} \big|_{\text{cool}}} \tag{19}
		\]
		\[
		\lambda = \frac{P}{\frac{dT}{dt} \big|_{\text{heat}} - \frac{dT}{dt} \big|_{\text{cool}}} \tag{20}
		\]

		\item При условии $T_{k,\text{heat}} = T_{k,\text{cool}} = T_k$, формулы упрощаются:
		\[
		C = \frac{P}{\left( \frac{dT}{dt} \big|_{\text{heat}} - \frac{dT}{dt} \big|_{\text{cool}} \right)} \tag{21}
		\]
		\[
		\lambda = \frac{P}{\left( \frac{dT}{dt} \big|_{\text{heat}} - \frac{dT}{dt} \big|_{\text{cool}} \right)} \cdot \frac{dT}{dt} \big|_{\text{cool}} \tag{22}
		\]
	\end{itemize}

	\item Сравним полученные значения $C$ и $\lambda$ по интегральному и дифференциальному методам. Сделаем вывод о точности и применимости методов.
\end{enumerate}

\section*{Вывод}

В ходе выполнения лабораторной работы:
\begin{itemize}
    \item Измерили кривые нагревания и охлаждения пустого калориметра и калориметра с образцами из железа и алюминия.
    \item Определили коэффициент теплоотдачи $\lambda$ и теплоемкость $C$ калориметра, используя интегральный метод на основе спрямления графиков $\ln(T - T_k)$.
    \item Вычислили теплоёмкости исследуемых образцов как разность между полной теплоёмкостью системы и теплоёмкостью пустого калориметра.
    \item Провели дополнительные расчёты по дифференциальным формулам, применяя их к «удобным точкам» графиков, и сравнили с результатами интегрального метода.
\end{itemize}

Полученные значения удельной теплоемкости исследуемых металлов оказались близки к табличным значениям. Наиболее точные результаты были достигнуты при стабильной комнатной температуре и достаточной длительности измерений, позволяющей исключить переходные режимы.

\vspace{1cm}

\section*{Контрольные вопросы}
\begin{enumerate}
    \item Как определяются удельные и молярные теплоемкости?
    \item Обоснуйте применение главной расчётной формулы работы. Запишите её в дифференциальной форме для процессов нагревания и охлаждения.
    \item Выведите формулу пересчёта кривых $R_{\text{heat}}(t)$, $R_{\text{cool}}(t)$ в $T_{\text{heat}}(t)$, $T_{\text{cool}}(t)$.
    \item Получите теоретические зависимости $T_{\text{heat}}(t)$ и $T_{\text{cool}}(t)$.
    \item При каких условиях корректно применять интегральный метод определения $C$ и $\lambda$? Что влияет на его точность?
    \item Зачем необходимо предварительно охлаждать калориметр на $2\text{–}5\,^\circ\text{C}$ ниже комнатной температуры?
    \item Опишите «удобные точки» на кривых нагревания и охлаждения, при которых корректно применять дифференциальные методы.
    \item Чему равна атомная теплоёмкость по классической теории? А молярная?
    \item Сравните точность интегрального и дифференциального методов определения теплоёмкости и коэффициента теплоотдачи.
\end{enumerate}

\end{document}



