\documentclass[a4paper,12pt]{article} % тип документа

% Поля страниц
\usepackage[left=2.5cm,right=2.5cm,
    top=2cm,bottom=2cm,bindingoffset=0cm]{geometry}
    
%Пакет дял таблиц   
\usepackage{multirow} 
    
%Отступ после заголовка    
\usepackage{indentfirst}


% Рисунки
\usepackage{floatrow,graphicx,calc}
\usepackage{wrapfig}

%%% Работа с картинками
\usepackage{graphicx}  % Для вставки рисунков
\graphicspath{{images/}}  % папки с картинками
\setlength\fboxsep{3pt} % Отступ рамки \fbox{} от рисунка
\setlength\fboxrule{1pt} % Толщина линий рамки \fbox{}
\usepackage{wrapfig} % Обтекание рисунков и таблиц текстом

% Создаём новый разделитель
\DeclareFloatSeparators{mysep}{\hspace{1cm}}

% Ссылки?
\usepackage{hyperref}
\usepackage[rgb]{xcolor}
\hypersetup{				% Гиперссылки
    colorlinks=true,       	% false: ссылки в рамках
	urlcolor=blue          % на URL
}


%  Русский язык
\usepackage[T2A]{fontenc}			% кодировка
\usepackage[utf8]{inputenc}			% кодировка исходного текста
\usepackage[english,russian]{babel}	% локализация и переносы

% Математика
\usepackage{amsmath,amsfonts,amssymb,amsthm,mathtools}

%%% Дополнительная работа с математикой
\usepackage{amsmath,amsfonts,amssymb,amsthm,mathtools} % AMS
\usepackage{icomma} % "Умная" запятая: $0,2$ --- число, $0, 2$ --- перечисление

% Что-то 
\usepackage{wasysym}


\begin{document}
\begin{center}
	\footnotesize{ФЕДЕРАЛЬНОЕ ГОСУДАРСТВЕННОЕ АВТОНОМНОЕ ОБРАЗОВАТЕЛЬНОЕ 			УЧРЕЖДЕНИЕ ВЫСШЕГО ОБРАЗОВАНИЯ}\\
	\footnotesize{МОСКОВСКИЙ ФИЗИКО-ТЕХНИЧЕСКИЙ ИНСТИТУТ\\(НАЦИОНАЛЬНЫЙ 			ИССЛЕДОВАТЕЛЬСКИЙ УНИВЕРСИТЕТ)}\\
	\footnotesize{ФИЗТЕХ-ШКОЛА ФИЗИКИ И ИССЛЕДОВАНИЙ им. ЛАНДАУ\\}
	\hfill \break
	\hfill \break
	\hfill \break
	\hfill \break
\end{center}

\begin{center}   
    \hfill \break
	\hfill \break
	\hfill \break
	\hfill \break    \hfill \break
	\hfill \break
	\hfill \break
	\hfill \break
    \hfill \break
    \hfill \break
	\hfill \break
	\large{Лабораторная работа № 2.4.1\\\textbf{Определение теплоты испарения жидкости}}\\
	\begin{flushright}
		Плотникова Анастасия Александровна\\
		Группа Б02-406
	\end{flushright}
	\hfill \break
	\hfill \break
	\hfill \break
\end{center}
\hfill \break
\hfill \break
\hfill \break
\hfill \break
\hfill \break
\hfill \break
\hfill \break
\hfill \break
\hfill \break
\hfill \break
\hfill \break
\hfill \break
\hfill \break
\begin{center}
	Долгопрудный, 2025 г.
\end{center}
\thispagestyle{empty}
\newpage
	\textbf{Цель работы:}\\ 1) измерение давления насыщенного пара жидкости при разной температуре;\\ 2) вычисление по полученным данным теплоты испарения с помощью уравения Клапейрона-Клаузиуса.
	\hfill \break
	
	\textbf{В работе используются:}\\ термостат,\\ герметический сосуд, заполненный водой,\\ отсчетный микроскоп.
	
\section{Теоретическая справка}

\textit{Испарение} — переход вещества из жидкого в газообразное состояние 
на свободной поверхности жидкости. 

Для выхода из жидкости молекуля должны совершить работу против сил молекулярного сцепления, 
внешнего давления, для чего должны обладать достаточной кинетической энергией. Переход части 
быстрых молекул в пар сопровождается потерями энергии, следовательно, и охлаждением жидкости.

\textit{Молярная теплота испарения (парообразования)} — количество теплоты, необходимое для 
изотермического испарения одного моля жидкости при внешнем давлении, равном упругости ее насыщенных паров.
    
В настоящей работе для определения теплоты испарения применен косвенный метод, 
основанный на формуле Клапейрона-Клаузиуса:
\begin{equation}
    \label{Klap}
    \frac{dP}{dT} = \frac{L}{T(V_2 - V_1)}
\end{equation}

\noindent $P$ -- давление насыщенного пара жидкости при температуре $T$,\\ 
$T$ -- абсолютная температура жидкости и пара,\\ 
$L$ — теплота испарения жидкости,\\ 
$V_2$ -- объем пара,\\ 
$V_1$ -- объем жидкости.

\begin{table}[h]
  \centering
  \includegraphics[scale = 0.55]{table_data.png}
  \caption{Табличные данные}
  \label{fig:table_data}
\end{table}

Найдя из опыта $dP/dT$, $T$, $V_2$ и $V_1$, можно определить $L$ путем расчета. 
Величины $L$, $V_2$ и $V_1$ в формуле (\ref{Klap}) должны относиться к одному и 
тому же количеству вещества; мы будем относить их к одному молю.

Измерения проводятся при давлениях ниже атмосферного.

Из табличных данных видно, что $V_1$ не превосходит $0.5\%$ от $V_2$. 

\medskip

Обратимся теперь к $V_2$, которое в дальнейшем будем обозначать $V$. 
Объем $V$ связан с давелнием и темературой уравнением Ван-дер-Ваальса:

\begin{equation}
    \label{Van-der}
    \left( P + \frac{a}{V^2}\right)(V - b) = RT
\end{equation}

Из рассмотрения таблицы (\ref{fig:table_data}) следует, что $b$ одного порядка с $V_1$. 
В уравнении Ван-дер-Ваальса величиной $b$ следует пренебречь. Пренебрежение членом $a/V^2$ 
по сравнению с $P$ вносит ошибку менее 3\%. 
При давлении ниже атмосферного ошибки становятся еще меньше. 

Таким образом, при давлениях ниже атмосферного уравнение Ван-дер-Ваальса для насыщенного пара 
мало отличается от уравнения Клапейрона. Положим поэтому

\begin{equation}
    \label{ideal}
    P = \frac{RT}{V}
\end{equation}

Подставляя (\ref{ideal}) в (\ref{Klap}) и разрешая уравение относительно $L$, получаем:

\begin{equation}
    \label{L}
    L = \frac{RT^2}{P}\frac{dP}{dT} = -R\frac{d(\mbox{ln }P)}{d(1/T)}
\end{equation}

Эта формула является окончательной.

\section{Экспериментальная установка}

Схема установки изображена на рисунке (\ref{fig:drawing2}). 

\begin{figure}[h]
  \centering
  \includegraphics[scale = 0.5]{drawing2.png}
  \caption{Схема установки для определения теплоты испарения}
  \label{fig:drawing2}
\end{figure}

\noindent 12. Емкость, заполненная водой. \\
13. Запаянный прибор с исследуемой жидкостью (14). 
Перед заполнением исследуемой жидкости воздух из запаянного прибора был удалён, 
так что над жидкостью находится только её насыщенный пар. \\
14. Исследуемая жидкость.\\
15. Ртутный манометр, соединённый с ёмкостью (13). \\
16. Отсчётного микроскопа, настраиваемого последовательно на нижний и верхний 
уровни столбика ртути манометра. \\
17. Штангенциркуль.

\medskip 

\bigskip
В эксперименте $T$ и $P$ определяются термостатом и разницей высот ртутных столбов, соответственно. 
Производные $\frac{dP}{dT}$ и $\frac{d(\mbox{ln }P)}{d(\frac1T)}$ можно найти из углового коэффициента 
касательной к графику $P(T)$ или как коэффициент наклона прямой на графике с осями ln $P$ и $\frac1T$.

\section{Ход эксперимента}

\begin{enumerate}
  \item При начальных условиях измерим разность уровней в ртутном U-образном манометре три раза с помощью микроскопа. 
  Для этого на одном из менисков установим значение нуля, и измерим модуль значения второго мениска. 
  С помощью термостата измерим температуру до и после начала этих трёх измерений. 
  Полученные значения занесём в таблицу (\ref{tab:init}).
  \item Включим термостат. 
  Будем подогревать воду в калориметре, повышая температуру с 20$^\circ$C до 40$^\circ$C. 
  Измерим температуру и разницу уровней менисков аналогичным образом.
  Результаты занесём в таблицу (\ref{tab:heating}).
  \item Рассчитаем среднюю температуру, среднюю разницу уровня ртути, давление, а также их погрешности. 
  Построим графики в координатах $T$, $P$ и в координатах $1/T$, $ln P$.
  \item Вычислим $L$, пользуясь формулой (\ref{L}) и данными, полученными сначала из одного, а потом из другого графика. 
  Сравним результаты, оценим ошибку измерений.
\end{enumerate}
\section{Результаты измерений}

$\rho_{Hg} = 13600$ кг/м$^3$

$P = 2\rho g\Delta h$, где $\Delta h = h_2 - h_1$ — разность высот столбов ртути

\medskip

Графики зависимости давления от температуры представлены на Рис.\ref{fig:graph1} и Рис.\ref{fig:graph2}.

Из графика (\ref{fig:graph1}) хорошо видно, что зависимость не линейная. 

\begin{table}[h]
    \caption{Начальные измерения}
    \centering
        \begin{tabular}{|c|c|c|c|c|c|c|c|}
    \hline $T_{0}^1$, $^\circ$C & $T_{0}^2$, $^\circ$C & $\Delta h_1$, мм & $\Delta h_2$, мм & $\Delta h_3$, мм & $\Delta h$, мм & $dh$, мм & $P_{0}$, Па  \\
    \hline  18.67 & 18.79 & 18.32 & 17.03 & 16.95 & 17.43 & 0.6 & 4651\\
    \hline
\end{tabular}
    \label{tab:init}
\end{table}

\begin{table}[h]
    \centering
    \begin{tabular}{|c|c|c|c|c|c|c|c|c|c|c|}
        \hline
        $T_1$, $^\circ C$ & $T_2$, $^\circ C$ & $T$, $^\circ C$ & $dT$, $^\circ C$ & $\varepsilon_T, \%$ & $h_1$, мм & $h_2$, мм  & $h_3$, мм  & $h$, мм  & $dh$, мм  & $\varepsilon_h, \%$ \\
        \hline
        20.09 & 20.00 & 20.05 & 0.05 & 0.22 & 16.17 & 17.41 & 16.82 & 16.80 & 0.40 & 2.5 \\
        21.00 & 21.01 & 21.00 & 0.01 & 0.02 & 18.20 & 18.60 & 19.00 & 18.60 & 0.30 & 1.4 \\
        22.07 & 22.04 & 22.06 & 0.01 & 0.07 & 19.75 & 19.66 & 19.33 & 19.58 & 0.20 & 0.9 \\
        23.02 & 23.04 & 23.03 & 0.01 & 0.04 & 21.26 & 21.28 & 21.35 & 21.30 & 0.04 & 0.2 \\
        23.99 & 24.01 & 24.00 & 0.01 & 0.04 & 22.92 & 22.53 & 23.18 & 22.88 & 0.20 & 1.0 \\
        25.02 & 25.03 & 25.03 & 0.01 & 0.02 & 24.64 & 24.40 & 24.42 & 24.49 & 0.10 & 0.4 \\
        25.99 & 26.01 & 26.00 & 0.01 & 0.04 & 25.45 & 25.82 & 25.95 & 25.74 & 0.20 & 0.8 \\
        27.00 & 27.00 & 27.00 & 0.00 & 0.00 & 27.79 & 27.30 & 27.06 & 27.38 & 0.30 & 1.0 \\
        28.04 & 27.99 & 28.02 & 0.03 & 0.09 & 28.97 & 28.79 & 28.80 & 28.85 & 0.08 & 0.3 \\
        29.01 & 29.01 & 29.01 & 0.00 & 0.00 & 30.82 & 30.65 & 30.68 & 30.72 & 0.07 & 0.2 \\
        \hline
    \end{tabular}
    \caption{Измерения при нагреве, их погрешности}
    \label{tab:measurements}
\end{table}

\begin{table}[h]
    \centering
    \begin{tabular}{|c|c|c|c||c|c|}
        \hline
        $P$, Па & $dP$, Па & $T$, $^\circ C$ & $dT$, $^\circ C$ & $\ln P$ & $1/T$, К$^{-1}$ \\
        \hline
        4482 & 110 & 20.05 & 0.05 & 8.41 & 0.00341 \\
        4963 & 70 & 21.00 & 0.01 & 8.51 & 0.00339 \\
        5224 & 40 & 22.06 & 0.01 & 8.56 & 0.00337 \\
        5682 & 9 & 23.03 & 0.01 & 8.64 & 0.00335 \\
        6104 & 60 & 24.00 & 0.01 & 8.72 & 0.00333 \\
        6533 & 30 & 25.03 & 0.01 & 8.78 & 0.00331 \\
        6868 & 50 & 26.00 & 0.01 & 8.84 & 0.00329 \\
        7306 & 70 & 27.00 & 0.00 & 8.90 & 0.00327 \\
        7698 & 20 & 28.02 & 0.03 & 8.95 & 0.00325 \\
        8196 & 20 & 29.01 & 0.00 & 9.01 & 0.00323 \\
        \hline
    \end{tabular}
    \caption{Таблица данных $P$, $dP$, $T$, $dT$ и вычисленных значений $\ln P$ и $1/T$}
    \label{tab:lnP_1T}
\end{table}

\begin{figure}
    \centering
    \includegraphics[scale = 0.75]{graph1.png}
    \caption{График зависимости $P(T)$}
    \label{fig:graph1}
  \end{figure}

\begin{figure}
    \centering
    \includegraphics[scale = 0.75]{graph2.png}
    \caption{График зависимости $\mbox{ln}P$ от $1/T_K$}
    \label{fig:graph2}
\end{figure}



\section{Вывод}
\end{document}
